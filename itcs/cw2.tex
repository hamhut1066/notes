\documentclass{article}
\usepackage{mathtools}
\usepackage{listings}
\usepackage{mathpartir}



\lstdefinelanguage{Lambda}{%
  morekeywords={%
    if,then,else,fix,letrec,in % keywords go here
  },%
  morekeywords={[2]nat,bool},   % types go here
  otherkeywords={:}, % operators go here
  literate={% replace strings with symbols
    {~>}{{$\rightarrow^\beta$}}{2}
    {->}{{$\to$}}{2}
    {lambda}{{$\lambda$}}{1}
  },
  basicstyle={\sffamily},
  keywordstyle={\bfseries},
  keywordstyle={[2]\itshape}, % style for types
  keepspaces,
  mathescape % optional
}[keywords,comments,strings]%


\title{ITCS Coursework 2}
\author{Hamish Hutchings  \\
  s1249759
  }

\date{\today}
% Hint: \title{what ever}, \author{who care} and \date{when ever} could stand
% before or after the \begin{document} command
% BUT the \maketitle command MUST come AFTER the \begin{document} command!
\begin{document}

\maketitle


\section*{Question 1}

\subsection*{a}
\label{subsec:a}

We already know that \textit{3-SAT} is in \textit{NP}. \textit{Exact 3-SAT} is a
special case of \textit{3-SAT}, such that \textit{Exact 3-SAT} \subset \textit{3-SAT}.

\paragraph{case 1}
let $\phi$ be an instance of \textit{3-SAT}. If every clause has three literals,
then $\phi$ is an instance of \textit{Exact 3-SAT}.

\paragraph{case 2}
 Construct $\phi'$ from
$\phi$, replacing all clauses with less than three literals with a clause that
repeats literals within itself (i.e. $A \vee B \rightarrow A \vee B \vee B$).

As we already know that \textit{3-SAT} is in \textit{NP}, we can say that
\textit{Exact 3-SAT} is in \textit{NP} as \textit{Exact 3-SAT} $\subset$ \textit{3-SAT}.

\subsection*{b}
\label{subsec:b}
only considering $x_i$ to be in $\{0,1\}$, we can define a routine $f(a, x)$
which returns true if $a$ and $x$ are the same, and false otherwise.

Examining a single \textit{3-product}
\[
  (a_1 - x_i)(a_2-x_j)(a_3-x_k)
\]
we can see that in order to satisfy the requirement of the evaluation of this
product to equal 0, either $(a_1 - x_i)$ or $(a_2 - x_j)$ or $(a_3 - x_k)$ must
equal 0.
For any of these sub-clauses to equal 0, the variables within must be equal. We
can now rewrite a sub-clause using the above defined function, which gives us an
\textit{Exact 3-SAT} problem as defined in the previous question.

\[
  f(a_1, x_i) \vee f(a_2, x_j) \vee f(a_3, x_k)
\]

expanding the above logical statement into a finite no. of \textit{3-product}'s,
we can again substitute each product form using the above technique.

\[
  [f(a_1, x_i) \vee f(a_2, x_j) \vee f(a_3, x_k)] \wedge \ldots  \wedge [f(a_n-2, x_a) \vee f(a_n-1, x_b) \vee f(a_n, x_c)]
\]

We can conclude that \textit{3PRODEQNS} problem is at least as hard as the
\textit{Exact 3-SAT} problem, and is therefore in \textit{NP}

\subsection*{c}
\label{subsec:c}

We can use \textit{CLIQUE} to solve \textit{INDSET}.

let $G(V,E)$ be a graph, and $k > 0$ be an instance of k-clique in $G$.
We can construct $G'(V',E'),k'$ where
\[
  V' = V \\
  E' = \{ \{u, v\} | u, v \in V, u \neq v and \{u,v\} \not \in E \} \\
  k' = k \\
\]

with $G'$, we can now solve for $k'$-clique. This will give us \textit{INDSET}
for $k$ in $G$. As \textit{CLIQUE} is \textit{NP}-complete, \textit{INDSET} is
also \textit{NP}-complete.

\paragraph{}
Above, we have defined a graph $G$, which we know has the a k-clique.
We then define a second graph $G'$ in terms of $G$, but with inverse edges.
This gives the second graph the satisfying property of all non-adjacent vertices
becoming adjacent. We already know how to solve for a sub-graph of k vertices
existing in a given graph (\textit{CLIQUE}). We further recognise that
\textit{INDSET} is a set of vertices in the inverse graph $G'$ that are adjacent
to each other. We can then show that in order to solve \textit{INDSET} we must
solve \textit{CLIQUE} which we already know to be \textit{NP}-complete.

\pagebreak[4]
\section*{Question 2}

\subsection*{a}
Evaluating the form described comes as such.
\begin{lstlisting}[language=Lambda] \\ \\
  (lambda m.lambda n.lambda f.lambda x.mf(nfx))(lambda f.lambda x.fx)(lambda f.lambda x.f(fx))
  ~> t[(lambda f.lambda x.fx)/m]

  (lambda n.lambda g.lambda y.(lambda f.lambda x.fx)g(ngy))(lambda f.lambda x.f(fx))
  ~> t[(lambda f.lambda x.f(fx))/n]

  (lambda g.lambda y.(lambda h.lambda z.hz)g((lambda f.lambda x.f(fx))gy))
\end{lstlisting}

This suggests that the lambda expression is evaluated by 'Call by name', This
allows for passing in expressions that would otherwise infinitely recurse, as
they are only evaluated when they are called. \\

This would mean that 'Call by value' would need another way to approach the
problem. As we cannot pass in an expression that may loop infinitely, we need to
express the problem in such a way that we can delay the evaluation of a value
until a later time, (I believe this is where Thunk's come in handy).

\subsection*{b}
\paragraph{Show that $Y'G = G(YG)$.}
\begin{lstlisting}[language=Lambda]
  YG = (lambda X.G(XX))(lambda X.G(XX))

  Y'G = (lambda F.(lambda X.XX)(lambda X.F(XX)))G

  ~>  (lambda X.XX)(lambda X.G(XX))

  ~>  (lambda X.G(XX))(lambda X.G(XX))

  ~>  G((lambda X.G(XX))(lambda X.G(XX)))

  = G(YG)
\end{lstlisting}

\paragraph{Show that $Y'$ reduces to $Y$ by an internal $\beta$-reduction.}
\begin{lstlisting}[language=Lambda]
  Y' = (lambda F.(lambda X.XX)(lambda X.F(XX)))

  ~>  (lambda F.(lambda X.F(XX))(lambda X.F(XX)))

  = Y
\end{lstlisting}

\subsection*{c}
Give a formal derivation for the type of the simple typed expression:
\begin{lstlisting}[language=Lambda]
  lambda f:nat -> nat.lambda x:nat. f(fx)
\end{lstlisting}

\paragraph{}
\[
  \inferrule{\inferrule{\inferrule{x:nat \vdash x: nat}
      {x:nat,f:nat \to nat \vdash fx: nat}}
    {x:nat, f: nat \to nat \vdash f(fx): nat}}
  {\lambda f:nat \to nat.\lambda x:nat. f(fx): (nat \to nat) \to nat \to nat}
\]

\linebreak
\subsection*{d}
\[
  \inferrule[FirstF]
  {\Gamma \vdash f: \pi \to \phi \\ \Gamma \vdash x: \pi}
  {\lambda f.\lambda x. fx: (\pi \to \phi) \to \pi \to \phi}

  \inferrule[SecondF]
  {\Gamma \vdash f: \alpha \to \beta \\ \inferrule
    {\Gamma, f: \alpha \to \beta \vdash f: \alpha \to \beta
    \\ \Gamma \vdash x: \alpha}
    {\Gamma \vdash fx: \beta}}
  {\lambda f.\lambda x. fx: (\alpha \to \beta) \to \alpha \to \beta}
\]

Above are the derivations for the two functions that are applied. I have done it
like this to make it slightly easier to read the final derivation.

\[
  \inferrule*
  {\Gamma \vdash m: (\pi \to \phi) \to \pi \to \phi
    \\ \Gamma \vdash f: \pi \to \phi
    \\ \inferrule*
    {\Gamma \vdash n: (\alpha \to \beta) \to \alpha \to \beta
      \\ \inferrule*
      {\Gamma \vdash f: \pi \to \phi \\ \Gamma \vdash x: \pi}
      {\Gamma \vdash fx: (\alpha \to \beta) \to \alpha}}
    {\Gamma \vdash nfx: \pi}}
  {\Gamma \vdash (\lambda m. \lambda n. \lambda t.\lambda x.mf(nfx))
    ((\pi \to \phi) \to \pi \to \phi)
    ((\alpha \to \beta) \to \alpha \to \beta)}
\]

We see from this that $\alpha = \pi$ and $\beta = \phi$, giving us the final
type:

\[
  (\lambda m: (\alpha \to \beta) \to \alpha \to \beta.\
  \lambda n: (\alpha \to \beta) \to \alpha \to \beta.
  \lambda f: \alpha \to \beta. \lamda x: \alpha . mf(nfx)))
\]
\[
  (\lambda f: \alpha \to \beta. \lambda x: \alpha. fx)
  (\lambda f: \alpha \to \beta. \lambda x: \alpha. f(fx))
\]

\pagebreak[4]
\section*{Question 3}

\subsection*{a}
\begin{lstlisting}[language=Lambda]
  letrec evenp = (lambda x.(if (zerop x)
    true
    ((lambda y. (if (evenp y) false (evenp (pred y))))(pred x))))
  in evenp(1)
\end{lstlisting}
\subsection*{b}
\begin{lstlisting}[language=Lambda]
  (lambda x.(if (zerop x)
    true
    ((lambda y. (if (evenp y) false (evenp (pred y))))(pred x))))1

  ~> (if (zerop 1)
      true
      ((lambda y. (if (evenp y) false (evenp (pred y))))(pred 1)))
\end{lstlisting}
\subsection*{c}
\[
  \begin{align}
  True: (\lambda t.\lambda f. t): \forall\alpha.\forall\beta.\alpha \to \beta \to
  \alpha \\
  False: (\lambda t.\lambda f. f): \forall\alpha.\forall\beta.\alpha \to \beta \to \beta
  \end{align}
\]
\subsection*{d}
\[
  \inferrule*
  {\inferrule*
    {\inferrule*
      {\inferrule*
        {\inferrule*
          {\Gamma, f: \alpha \to \beta \vdash x: \alpha}
          {\Gamma, f: \alpha \to \beta \vdash f
            \\ \Gamma, f: \alpha \to \beta \vdash fx}}
        {\Gamma, f: \alpha \to \beta \vdash f(fx): \beta}}
      {\Gamma, f:\alpha \to \beta \vdash (\lambda x. f(fx)): \alpha \to \beta}}
    {\Gamma \vdash (\lambda f.\lambda x.f(fx)): \forall \alpha.\forall \beta.(\alpha \to
      \beta) \to \alpha \to \beta
      \\ \Gamma \vdash succ: nat \to nat
      \\ \Gamma \vdash 3: nat}}
  {(\lambda f.\lambda x.f(fx))(suc)(3)}
\]

Above is a rough derivation of types.
The final formula is described below:

\[
  (\lambda f:(\forall\alpha.\forall\beta.\alpha \to \beta)
  .\lambda x:\forall\alpha. \alpha. f(fx))(suc: nat \to nat)(3:nat)
\]

The variables are unified when the function \textit{succ} is passed into the
lambda function. the abstract types $\alpha$ and $\beta$ are specialised to the
\textit{nat} type.



\end{document}
